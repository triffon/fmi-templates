% опции на класа:
% bulgarian : български език (по подразбиране)
% english   : английски език
% assocprof : доцент (по подразбиране)
% prof      : професор
\documentclass{fmirsacad}

\field{\emph{професионално направление}}

\stategazette{\emph{брой на държавен вестник}}{\emph{дата на държавен вестник}}

\reviewer{\emph{академична длъжност, научна степен, име, презиме, фамилия - месторабота на рецензента}}{\emph{професионално направление на рецензента}}

\order{\emph{номер на заповедта}}{\emph{дата на заповедта}}

% ако има няколко кандидата, може да има няколко команди \addcandidate
\addcandidate{\emph{академична длъжност, научна степен, име, презиме, фамилия}}{\emph{научна организация}}

% да се смени review със statement, ако е становище вместо рецензия
\begin{review}

  % по една среда за всяка кандидатура
  % да се смени positive с negative, ако заключението е отрицателно
  % вътре в текста името на кандидата може да се реферира с \candidate или \Candidate (с първа главна буква)
  \begin{candidature}{positive}

    \papers{1}
    % могат да се добавят и други видове работи
    % \studia{\emph{брой}}
    % \monographs{\emph{брой}}
    % \books{\emph{брой}}
    % \patents{\emph{брой}}
    % \textbooks{\emph{брой}}

    % да се разкоментира при нужда
    % \otherdocs{\emph{брой на другите документи, ако са представени такива (във вид на служебни бележки и удостоверения от работодател, ръководител на проект, финансираща организация или възложител на проект, референции и отзиви, награди и други подходящи доказателства)}}{\emph{Бележки и коментар по документите.}}

    \begin{generaldocscomment}
      \emph{Общи коментари по предадените документи.}
    \end{generaldocscomment}

    \begin{candidatedata}
      \emph{Кратки професионални и биографични данни за кандидата.}
    \end{candidatedata}

    \begin{resultsreview}
      \emph{Дава се оценка на научните резултати на кандидата. В кои научни области и по кои проблеми е работил и продължава да работи кандидатът.}
    \end{resultsreview}

    \begin{teachingreview}
      \emph{Оценка на учебно-педагогическа дейност на кандидата.}
    \end{teachingreview}

    \begin{scientificreview}
      \emph{Оценяват се научните и научно-приложните приноси на кандидата, като се заяви ясно какъв е характерът им: нови теории, хипотези, методи и др.; обогатяване на съществуващи знания; приложение на научни постижения в практиката. Отражение на резултатите на кандидата в трудовете на други автори. Числови показатели – цитати, импакт-фактор и др. При колективни публикации да се отрази приносът на кандидата.}
    \end{scientificreview}

    \begin{recommendations}
      \emph{Посочват се критични бележки, отправени към рецензираните трудове по отношение на: постановка; анализи и обобщения; методично равнище; точност и пълнота на резултатите; литературна осведоменост.}
    \end{recommendations}

    \begin{personalcomments}
      \emph{Лични впечатления от кандидата}
    \end{personalcomments}
  \end{candidature}
\end{review}
