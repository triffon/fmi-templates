\documentclass[assocprof,bulgarian]{fmirsacad}

\field{4.6 Информатика и компютърни науки}

\organisation{Софийски университет „Св. Климент Охридски“}
\org{СУ}
\institute{Факултет по математика и информатика}
\inst{ФМИ}

\stategazette{52}{17.07.2019 г.}

\reviewer{проф. д-р Иван Иванов Иванов --- ФМИ}{Информатика и компютърни науки}

\order{12}{20.09.2019 г.}

\addcandidate{гл. ас. д-р Петя Петрова Петрова}{ФМИ}
\addcandidate{гл. ас. д-р Ринсуинд Магьосника}{Невидим Университет}

\begin{statement}

\begin{candidature}{positive}
  \papers{10}
  \monographs{1}
  \textbooks{1}

  \otherdocs2{Кандидатът е представил две препоръки.}

  \begin{candidatedata}
    Кандидатът \candidate е завършила ОКС "`магистър по Информатика"' във ФМИ на СУ през 2000 г.
  \end{candidatedata}

  \begin{resultsreview}
    Кандидатът има сериозни научни резултати в областите изкуствен интелект и структури от данни и алгоритми.
  \end{resultsreview}

  \begin{teachingreview}
    Кандидатът \candidate има 5 год. преподавателска дейност във ФМИ на СУ.
  \end{teachingreview}

  \begin{scientificreview}
    Разработките на \candidate имат приложно-научен характер и дефинират нови методи за прилагане на стандартни алгоритми над структури от данни в областта на изкуствения интелект.

    Представени са свидетелства за 20 цитата, а 3 от работите са с импакт фактор. Всички публикации са самостоятелни.
  \end{scientificreview}

  \begin{recommendations}
    Бих препоръчал на кандидата да публикува резултатите си в издания посветени на изкуствен интелект, а не само такива фокусирани над алгоритмичния аспект.
  \end{recommendations}

  \begin{personalcomments}
    Познавам \candidate от 8 години, когато успешно положи изпит за асистент в катедрата.
  \end{personalcomments}
\end{candidature}

\begin{candidature}{negative}
  \papers{3}
  \books{2}

  \otherdocs1{Кандидатът е представил една служебна бележка от бивш работодател.}

  \begin{generaldocscomment}
    Документите са оформени небрежно и представени на пергамент.
  \end{generaldocscomment}

  \begin{candidatedata}
    Кандидатът \candidate е представил диплома за завършен ОКС "`магистър по Окултни науки"' в Невидимия университет на Анкх-Морпорк през 2005 г., която не е призната съгласно вътрешния правилник на СУ.
  \end{candidatedata}

  \begin{resultsreview}
    Научните резултати на кандидата са основно в областта на окултните науки.
  \end{resultsreview}

  \begin{teachingreview}
    Кандидатът \candidate има 1 год. преподавателска дейност във ФМИ на СУ.
  \end{teachingreview}

  \begin{scientificreview}
    Разработките на \candidate имат теоретичен характер и разглеждат приложенията на информационните технологии в окултните науки.

    Представени са свидетелства за 10 цитата, а само 1 от работите има импакт фактор. Всички публикации са съвместни с научния ръководител на кандидата, M.~Ridcully.
  \end{scientificreview}

  \begin{recommendations}
    Бих препоръчал на кандидата да задълбочи изследванията си в областта на теоретичната информатика от гледна точка на окултните науки.
  \end{recommendations}

  \begin{personalcomments}
    Нямам лични впечатления от кандидата.
  \end{personalcomments}
\end{candidature}

\end{statement}
