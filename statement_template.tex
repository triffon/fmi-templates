\documentclass[a4paper]{report}

\usepackage[margin=2cm]{geometry}
\usepackage{enumitem}
\usepackage[utf8]{inputenc}
\usepackage[T2A]{fontenc}
\usepackage[bulgarian]{babel}

\def\assocprof{доцент}
\def\prof{професор}
\def\position{\ldots}          % TODO: \assocprof или \prof

\def\profarea{\ldots}          % TODO: професионално направление
\def\statepaperno{\ldots}      % TODO: брой на държавен вестник
\def\statepaperdate{\ldots}    % TODO: дата на държавен вестник
\def\reviewer{\ldots}          % TODO: академична длъжност, научна степен, име, презиме, фамилия - месторабота на рецензента
\def\reviewerprofarea{\ldots}  % TODO: професионално направление на рецензента
\def\orderno{\ldots}           % TODO: номер на заповед
\def\orderdate{\ldots}         % TODO: дата на заповед

\def\singlecandidate{е подал документи \textbf{единствен кандидат}}
\def\manycandidates{са подали документи \textbg{следните кандидати}}
\def\candidatenumber{\ldots}        % TODO: \singlecandidate или \manycandidates
\newcommand{\candidate}[1]{\item #1}
\newenvironment{candidates}{\begin{itemize}}{\end{itemize}}
\def\candidatename{\ldots}                % TODO: име на кандидата

\def\titlecount{\ldots}                   % TODO: брой на статиите, представени за конкурса
\def\otherdocscount{\ldots}               % TODO: брой на другите документи (във вид на служебни бележки и удостоверения от работодател, ръководител на проект, финансираща организация или възложител на проект, референции и отзиви, награди и други подходящи доказателства)

\def\conclusion{1}                        % TODO: 1 за положителна оценка, 0 за отрицателна
\def\confirm{%
  \if\conclusion0%
  не \fi%
  потвърждавам}
\def\evaluation{%
  \if\conclusion0%
  отрицателна%
  \else положителна%
  \fi}
\def\recommendation{%
  \if\conclusion0%
  не \fi%
  препоръчвам}

\newcommand{\titletype}[2]{
  \ifx\testforempty#2\testforempty
  \relax
  \else
  #2 #1,
  \fi}
\newcommand{\papers}{\titletype{публикации в български и чуждестранни научни издания и научни форуми}}
\newcommand{\studia}{\titletype{студии}}
\newcommand{\monographs}{\titletype{монографии}}
\newcommand{\books}{\titletype{книги}}
\newcommand{\patents}{\titletype{свидетелства и патенти}}
\newcommand{\textbooks}{\titletype{учебници и учебни пособия}}

\begin{document}

\begin{center}
  \Large\bfseries\MakeUppercase{Становище}\\
  % FIXME: разстояние между редовете
  \small
  по конкурс за заемане на академична длъжност\\
  ``\position''\\
  в професионално направление \position,\\
  за нуждите на Софийски университет ``Св. Климент Охридски'' (СУ),\\
  Факултет по математика и информатика (ФМИ),\\
  обявен в ДВ бр. \statepaperno\ от \statepaperdate\ и на интернет страниците на ФМИ и СУ
\end{center}

Становището е изготвено от \reviewer, \reviewerprofarea, в качеството му на член на научното жури по конкурса съгласно Заповед № \orderno/\orderdate на Ректора на Софийския университет.

За участие в обявения конкурс \candidatenumber:

\begin{candidates}
\candidate{\ldots} % TODO: списък с кандидати с академична длъжност, научна степен, име, презиме, фамилия, научна организация
\end{candidates}

\begin{enumerate}[label=\textbf{\Roman*.}]
\item \textbf{Общо описание на представените материали\\
    За всеки от кандидатите се дава информация по точки от 1 до 8:}

  \begin{enumerate}[label=\textbf{\arabic*.}]
  \item \textbf{Данни за кандидатурата}

    Представените по конкурса документи от кандидата съответстват на изискванията на ЗРАСРБ, ППЗРАСРБ и Правилника за условията и реда за придобиване на научни степени и заемане на академични длъжности в СУ ``Св. Климент Охридски'' (ПУРПНСЗАДСУ).

    За участие в конкурса кандидатът \candidatename\ е представил списък от общо \titlecount\ заглавия, в т.ч
    % TODO: да се попълнят бройките
    \papers{\ldots}
    \studia{}
    \monographs{}
    \books{}
    \patents{}
    \textbooks{}.

    Представени са \otherdocscount\ на брой други документи, покрепящи постиженията на кандидата.
    
    % TODO: Бележки и коментар по документите.

    
  \item \textbf{Данни за кандидата}
    % TODO: Кратки професионални и биографични данни за кандидата.

  \item \textbf{Обща характеристика на научните трудове и постижения на кандидата}

    % TODO: Дава се оценка на научните резултати на кандидата. В кои научни области и по кои проблеми е работил и продължава да работи кандидата.

    След подробен преглед на представените научни трудове установих, че:
    \begin{enumerate}[label=\alph*)]
    \item научните трудове отговарят на минималните национални изисквания (по чл. 2б, ал. 2 и 3 на ЗРАСРБ) и съответно на допълнителните изисквания на СУ „Св. Климент Охридски“ за заемане на академичната длъжност ``\position'' в научната област и професионално направление на конкурса;
    \item представените от кандидата научни трудове не повтарят такива от предишни процедури за придобиване на научно звание и академична длъжност;
    \item няма доказано по законоустановения ред плагиатство в представените по конкурса научни трудове.
    \end{enumerate}
    
  \item \textbf{Характеристика и оценка на преподавателската дейност на кандидата}

    % TODO: Оценка на учебно-педагогическа дейност на кандидата.

  \item \textbf{Съдържателен анализ на научните и научно-приложните постижения на кандидата съдържащи се в материалите за участие в конкурса}

    % TODO: Оценяват се научните и научно-приложните приноси на кандидата, като се заяви ясно какъв е характерът им: нови теории, хипотези, методи и др.; обогатяване на съществуващи знания; приложение на научни постижения в практиката. Отражение на резултатите на кандидата в трудовете на други автори. Числови показатели – цитати, импакт-фактор и др. При колективни публикации да се отрази приносът на кандидата.
    
  \item \textbf{Критични бележки и препоръки}

    % TODO: Посочват се критични бележки, отправени към рецензираните трудове по отношение на: постановка; анализи и обобщения; методично равнище; точност и пълнота на резултатите; литературна осведоменост.

  \item \textbf{Посочват се критични бележки, отправени към рецензираните трудове по отношение на: постановка; анализи и обобщения; методично равнище; точност и пълнота на резултатите; литературна осведоменост.}
    
  \item \textbf{Заключение за кандидатурата}
    
    След като се запознах с представените в конкурса материали и научни трудове и въз основа на направения анализ на тяхната значимост и съдържащи се в тях научни и научно-приложни приноси, \textbf{\confirm}, че научните постижения отговарят на изискванията на ЗРАСРБ, Правилника за приложението му и съответния Правилник на СУ „Св. Климент Охридски“ за заемане от кандидата на академичната длъжност ``\position'' в научната област и професионално направление на конкурса. В частност кандидатът удовлетворява минималните национални изисквания в професионалното направление и не е установено плагиатство в представените по конкурса научни трудове.
    
    Давам своята \textbf{\evaluation} оценка на кандидатурата.
  \end{enumerate}
  
\item \textbf{\MakeUppercase{Общо заключение}}

  Въз основа на гореизложеното, \textbf{\recommendation} на научното жури да предложи на компетентния орган по избора на Факултета по математика и информатика при СУ „Св. Климент Охридски“ да избере \candidatename\ да заеме академичната длъжност ``\position'' в професионално направление \profarea.
\end{enumerate}

\vspace{5ex}

\begin{tabular}{l@{\hspace{18em}}c}
  \@date & Изготвил становището: \makebox[15em]\hrulefill\\
  & \reviewer
\end{tabular}

\end{document}